\chaptertoc{Introduction}
	Selon une étude de Statista sur l'utilisation du téléphone mobile dans le monde, le nombre d'utilisateur de téléphone mobile dans le monde devrait dépasser la barre des cinq milliards en 2019 et d'ici 2020, le nombre d'utilisateurs de smartphone devrait atteindre 2,87 milliards d'individus. Selon une autre étude réalisée par \og Internetworldstats \fg{,}  au 30 juin 2019, le nombre d'utilisateurs d'internet dans le monde a atteint les $4.536.248.808$ soit $58 \% $ de la population mondiale dont environ 3,986 milliards\footnote{Etude réalisée par Hootsuite} sont des connexions réalisées à partir des téléphones mobiles.
	
	De ces études on peut en déduire clairement qu'il est primordial pour une entreprise aujourd'hui de proposer ses services internet pour les appareils mobiles  qui  représentent $88\% $ de l'accès à internet. 
	
	De nos jours, le développement pour mobile a pris beaucoup d'ampleur  et de plus en plus d'entreprises l'ont adopté pour leurs produits car c'est un moyen de créer des services innovants, d'améliorer la communication et d'augmenter leur productivité.
	
	Du 02 septembre 2019 au 07 février 2020 (5 mois 5 jours), j'ai effectué un stage assistant ingénieur au sein de l'entreprise \textbf{Luxembourg Online SA} (située à Luxembourg). Au cours de ce stage dans le département informatique, j'ai pu mettre mes compétences de développeurs logiciels pour développer plusieurs applications mobiles pour les systèmes d'exploitations Android et iOS.
	
	Luxembourg Online est l'un des principaux opérateurs luxembourgeois de télécommunications. La société est spécialisée dans la fourniture d'accès internet, la téléphonie fixe, mobile, la télévision, le développement de réseaux et d'applications informatiques.
	
	Le service de l'entreprise qui m'a accueillit pour mon stage est le service informatique qui est dirigé par mon maître de stage M. Paul Retter. Mon stage a consisté essentiellement en le développement de plusieurs applications mobiles pour les platformes Android et iOS.
	
	Plus largement, ce stage a été l'opportunité pour moi non seulement d'approfondir mes connaissances dans le développement pour Android et de travailler sur une application grand public mais aussi d'apprendre à développer des applications pour la platforme iOS d'apple et d'approfondir mes connaissances dans le domaine.
	
	Au delà d'enrichir mes connaissances en développement logiciel, ce stage m'a permis de comprendre certains aspect du développement notamment la programmation réactive, l'architecture logicielle,  et de le mettre en pratique.
	
	Dans l'optique de rendre compte de manière fidèle des 5 mois passés au sein de la société  Luxembourg Online, il apparaît logique de présenter à titre préalable de l'état actuel des solutions internet et mobile de l'entreprise, ensuite envisager le cadre du stage : la culture d'entreprise dans la société Luxembourg Online et son apport dans la méthode de travail et la productivité et enfin préciser les différentes missions et tâches que j'ai pu efectuer au sein du service informatique, et les nombreux apports que j'ai pu en tirer.