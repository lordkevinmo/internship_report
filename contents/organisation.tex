\chapter{Organisation du stage}
\section*{Travail réalisé}
Le sujet défini avant le début de mon stage est intitulé \og Développement d'applications sous \gls{Android} et \gls{iOS} \fg{.}  

A mon arrivé dans l'entreprise j'ai eu une réunion avec mon tuteur de stage, réunion au cours de laquelle il m'a été expliqué concrètement le travail que j'effectuerai durant la période de mon stage.

L'objectif de mon stage est de développer deux applications respectivement nommé \textbf{Smart Home Viewer} en version \gls{Android} puis \gls{iOS} et \textbf{Resto}\footnote{Nom de code de l'application. Le nom de marque n'étant pas encore choisi.}
\paragraph*{}
Le développement de l'application Smart Home Viewer avait été commencé par un stagiaire mais il n'a pas pu le terminé avant son départ. Mon travail a consisté donc à prendre en main le travail incomplet, de le terminer, de corriger les éventuels bogues après les tests et ensuite de commencer à développer l'application Resto.

Le suivi du développement est réalisé de manière hebdomadaire. Sur une semaine, je réalisais des cycles de développement court pendant lequel, je travaillais personnellement en Agile avec la méthode Kanban. A la fin de la semaine, je remplissais une fiche des travaux réalisés que je transmettais via la Gestionnaire des Ressources Humaines à mon tuteur. Je sortais aussi par la même occasion une \textit{release} du projet que des testeurs externes à mon service mais interne à l'entreprise pouvait tester et faire des retours d'informations.

J'ai réussi à produire une \textit{release} finale à la fin du mois de Septembre pour l'application Smart Home Viewer.
\paragraph*{}

Au début du mois d'Octobre j'ai eu une nouvelle réunion majeure avec mon tuteur de stage et une équipe de développement composée de développeur Back-end, d'un administrateur de base de donnée et d'un développeur \gls{Android} pour lancement du développement de l'application Resto.

Pour cette application, je me suis occupé du développement de la version \gls{iOS}. L'application est composée d'une partie cliente et d'une partie Pro. Ma tâche est d'arrivée à produire une version Bêta à la fin du mois d'octobre et de faire ensuite l'application vers une version majeure vers la fin du mois de Novembre. Ensuite il sera question de travailler sur la version Pro jusqu'à la fin du stage.

Pour cette application, j'ai travaillé directement en collaboration avec le développeur \gls{Android}, le développeur Backend et un designer \gls{UI}.
\paragraph*{}
Dans la suite de ce document, je présenterai chacune des applications que j'ai développé, leurs spécificités, les difficultés que j'ai rencontré et les solutions que j'ai trouvé. Je ferai à la fin un bilan sur la totalité du travail 