\chaptertoc{Conclusion}
\p{Entreprise}
L'entreprise Luxembourg Online est exceptionnel dans le sens où l'apprentissage est autorisé au cours du stage. Ainsi avant mon arrivé dans l'entreprise, je ne savais pas codé pour les plateformes de Apple mais la possibilté m'a été laissé d'apprendre la technologie afin de travailler de manière autonome. Une des particularité que j'ai aimé de l'entreprise est qu'elle développe elle même presque tous les outils utilisés. Cette façon de faire à pousser ma curiosité à explorer des domaines que je n'aurai pas exploré si je n'avais pas fait ce stage
\paragraph*{}
\p{Travail réalisé}
Je suis content du travail que j'ai réalisé au cours de mon stage. Le sujet initial étant le développement d'application sous android et iOS, j'était venu avec l'handicap de ne pas savoir comment développer pour les plateformes iOS, mais j'ai su vite apprendre et produire les deux premières applications le premier mois de stage. J'ai ainsi pu travailler sur deux autres applications à la suite.
\paragraph*{}
\p{Expériences}
J'ai appris énormément au cours de ce stage et j'ai appris beaucoup sur moi même. Le travail que j'ai réalisé m'a permis d'améliorer mon niveau en conception et architecture logicielle, il m'a permis d'apprendre à apprendre et assimiler en moins de 48h un langage de programmation et en 1 semaine une technologie.
\paragraph*{}
Ce stage a été pour moi l'occasion de travailler sur des sujets assez intéressant et des applications grand public. Les spécificités de ces types d'applications sont souvent différentes des projets que nous réalisons dans le cadre scolaire. Ainsi j'ai appris à faire une application qui sera utilisé par un grand nombre d'utilisateur. Pour ce faire, il faut faire au préalable une conception approfondie de l'application avant de commencer le développement. J'ai ainsi à travers ce stage amélioré mes compétences en développement logiciel.